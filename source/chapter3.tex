\chapter{Methodology}

\begin{comment}
    Outline of the Methodology:
    - Introduction
    - Verification of Deadlock-Resolving RDLT
        - Contraction Path Generation (for identificaiton of deadlocks)
        - identification of escape contraction paths
        - Matrix-based algorithm for verification of Deadlock-Resolving RDLT
    - Loop-Safeness and Safeness Verification (Matrix-based)
    - Weakened JOIN-Safeness Verification (Matrix-based)
    - Weak Soundness of RDLT Verification (Matrix-based)
    - Easy Soundness of RDLT Verification (Matrix-based)

\end{comment}

This section introduces a matrix-based implementation of the verfication of weak and easy soundness of RDLT. The general algorithm for these verification is based on the formalizations of \cite{Ramirez2024} in terms of structural verifications of the said soundness properties. 

Firstly, on weak soundness, \cite{Ramirez2024} proposed WRSVA in which Deadlock-Tolerance is verified in order to verify weather or not an RDLT is weak sound. As explained in the related literatures, deadlock-tolerance refers to an RDLT being deadlock-resolving, having all NCAs loop-safe, CAs safe, and having all its split-join pairs as weakened JOIN-safe. All of these properties are verified through the use of matrices and their corresponding operations. The proofs and test-cases are also outlined for each verification of these mentioned properties. 

Similarly, easy soundness verification as formalized by \cite{Ramirez2024} is also implemented using matrix representation and operations. Its property of having an option to complete is verified by finding a contraction path from the source of R to the sink. Corresponding proofs and test-cases are also included to this proposed method. 

\subsection{Verification of Deadlock-Resolving RDLT}
% I think I should include a reference of the theorems from Ramirez' regarding his definitions. 
Theorem 3.4.3 in \cite{Ramirez2024} explains that an RDLT $R$ is of weak soundness if and only if both level-1 and level-2 vertex simplifications of $R$ are deadlock-tolerance. The paper defines deadlock-tolerance as an RDLT where every $NCA$ are loop-safe, every $CA$ are safe, $R$ is weakened JOIN-safe, and is deadlock-resolving.

For the verification of whether $R$ is deadlock-resolving, the matrix representation of which includes the enumeration of relevant deadlocks in $R$, and the identification of the existence of escape contraction paths from the parent of each deadlocks. The existence of an escape contraction paths is what ensures proper termination albeit the existence of deadlocks within an RDLT, removing the liveness property. Therefore, such RDLT is of weak soundness.

\subsubsection*{Identification of Deadlocks}
On the identification deadlocks, a method is proposed which involves the contraction \cite{Malinao2017} of the expanded vertex simplified RDLT $R_i$. This contraction algorithm as proposed by Malinao in here dissertation checks the c-verifiability of the RDLT \cite{Malinao2017}. Two adjacent vertices are contractable or can be merged when the arc currently being checked has contraint/s which is a subset of all the contraints coming towards the other vertex the algorithm is trying to contract. The final contraction shows the contraction path, which is a path of vertcies, that are reachable through an activity, hence through activity extraction, as well. That said, no unreachable vertices will be included in the contracted vertices, hence deadlocks, as defined in \cite{Ramirez2024}, can be identified.

This paper implements a matrix-based contraction algorithm adopted from the modified contraction algorithm (MCA) introduced in \cite{malinao2024model}. The following sections define the necessary matrices and algorithms to achieve a research objective. 

% Matrices 

\begin{defn} \textbf{Adjacency Matrix $RV_{adj}$} \\
Let $R_i = \bigl(V_i, E_i,\Sigma_i, C_i, L_i, M_i\bigr)$ be an expanded vertex simplification of an RDLT $R$. The $n \times n$ matrix of a directional graph, where $n = |V_i|$ and the rows and columns correspond to the vertex set $V_i = \{\,v_1, v_2, \ldots, v_n\}$ at time step $t$, is defined as

\[
  RV_{adj}^{t} \;=\;
  \begin{pmatrix}
    RV_{\mathrm{adj}}(v_1,v_1) & RV_{\mathrm{adj}}(v_1,v_2) & \cdots & RV_{\mathrm{adj}}(v_1,v_n) \\[6pt]
    RV_{\mathrm{adj}}(v_2,v_1) & RV_{\mathrm{adj}}(v_2,v_2) & \cdots & RV_{\mathrm{adj}}(v_2,v_n) \\[6pt]
    \vdots                    & \vdots                    & \ddots & \vdots                    \\[6pt]
    RV_{\mathrm{adj}}(v_n,v_1) & RV_{\mathrm{adj}}(v_n,v_2) & \cdots & RV_{\mathrm{adj}}(v_n,v_n)
  \end{pmatrix}.
\]

Each entry $RV_{adj}(x,y)$ represents the number of distinct arcs from $x$ to $y$, given by
\[
  RV_{adj}(x,y) \;=\;
  \begin{cases}
    RV_{adj}(x,y) \in \mathbb{N}, & \text{if } (x,y)\,\in E_i \text{ in } R_i, \\[5pt]
    0,     & \text{otherwise}.
  \end{cases}
\]
\end{defn}

\begin{defn} \textbf{Constraints Matrix $RV_C$} \\
We similarly define the matrix $RV_c$. The $n \times n$ matrix of a directional graph, where $n = |V_i|$ and the rows and columns correspond to the vertex set $V_i = \{\,v_1, v_2, \ldots, v_n\}$ at time step $t$, is defined as
\[
   RV_{C}^{t} \;=\;
  \begin{pmatrix}
    RV_c(v_1,v_1) & RV_c(v_1,v_2) & \cdots & RV_c(v_1,v_n) \\[6pt]
    RV_c(v_2,v_1) & RV_c(v_2,v_2) & \cdots & RV_c(v_2,v_n) \\[6pt]
    \vdots        & \vdots        & \ddots & \vdots        \\[6pt]
    RV_c(v_n,v_1) & RV_c(v_n,v_2) & \cdots & RV_c(v_n,v_n)
  \end{pmatrix}.
\]
Each entry $RV_{C}^{t}(x,y)$ represents the corresponding C-attribute of the arcs from $x$ to $y$ in the Adjacency Matrix, given by
\[
  RV_{C}^{t}(x,y) \;=\;
  \begin{cases}
    C(x,y)\in \Sigma \cup \{\epsilon\}, & \text{if } (x,y)\in E_i \text{ in } R_i,\\[5pt]
    \varnothing,                                  & \text{otherwise}.
  \end{cases}
\]
\end{defn}

\newpage
 
\begin{algorithm}[H]
\caption{\textbf{Matrix-based Contraction Path Generation}}
\label{alg:contractionPathGeneration}
\textbf{Given} RDLT R; Pre-processing Steps:

\textbf{Input}: Expanded vertex simplification $R_i$ of RDLT R

\textbf{Output}: Contraction Path P

\textbf{Matrices}: $RV^{t}_{\text{adj}}$ and $RV^{t}_{\text{C}}$

    \begin{algorithmic}[1]

    \State Initialize C-Attribute Matrix $RV^{0}_{\text{C}}$ of $R_i$
    \State Let $s' \in V_i$ be the source and $f' \in V_i$ be the sink
    \State Let $x = s'$
    \State Initialize $P = \{x\}$
    \State Let $t = 1$

    % \While{$P$ does not contain $f'$}
    % \While{\exists\, x \in X}
    % \While{\exists $y$ \in $V_i$}
    \While{$\exists y \in V_i \mid RV^{t-1}_{\text{adj}}(x,y) \geq 1$}

        \State $\mathcal{Y} \leftarrow \{ y \in V_i \mid RV^{t-1}_{\text{adj}}(x,y) \geq 1 \}$
        \State Select any $y \in \mathcal{Y}$
        \State Let $\text{LHS} = RV^{t-1}_{C}(x,y) \cup \{\epsilon\}$
        \State $\mathcal{U} \leftarrow \{ u \in V_i \mid u \neq x \wedge (RV^{t-1}_{\text{adj}}(u,y) \geq 1) \}$
        \State Let $\text{RHS} = \bigcup_{u \in \mathcal{U}} RV^{t-1}_{C}(u,y)$
        \If {$\text{LHS} \supseteq \text{RHS}$}
            \State Update $RV^{t-1}_{C}(u,y) = \epsilon, \forall (u,y) \in E_i, u \neq x$
            \ForAll {$u \in \mathcal{U}$}
                \State $RV^{t-1}_{C}(u,y) = \epsilon$
            \EndFor
            \State Let $z = x \land y$
            \State Let $z = xy$ = Matrix Addition of rows (columns) $x$ and $y$ in $RV^{t-1}_{\text{adj}}$
            \ForAll {$w \in V_i$}
                \State RowMerge\_Adj: $RV^{t}_{\text{adj}}(z,w) = RV^{t-1}_{\text{adj}}(x,w) + RV^{t-1}_{\text{adj}}(y,w)$
                \State ColMerge\_Adj: $RV^{t}_{\text{adj}}(w,z) = RV^{t-1}_{\text{adj}}(w,x) + RV^{t-1}_{\text{adj}}(w,y)$
            \EndFor
            \State Let $z = xy$ = Element-wise Set Union of rows (columns) $x$ and $y$ in $RV^{t-1}_{C}$
            \ForAll {$w \in V_i$}
                \State RowMerge\_C: $RV^{t}_{C}(z,w) = RV^{t-1}_{C}(x,w) \cup RV^{t-1}_{C}(y,w)$
                \State ColMerge\_C: $RV^{t}_{C}(w,z) = RV^{t-1}_{C}(w,x) \cup RV^{t-1}_{C}(w,y)$
            \EndFor
            \State $V_i = (V_i \setminus \{x,y\}) \cup \{z\}$
            \State Create $RV^{t}_{\text{adj}}$ as an $m \times m$ matrix where $m = n - t$ and as the submatrix of $RV^{t-1}_{\text{adj}}$ with rows and columns indexed by updated vertex set $V_i$ of $R_i$
            \State Create $RV^{t}_{C}$ as an $m \times m$ matrix where $m = n - t$ and as the submatrix of $RV^{t-1}_{C}$ with rows and columns indexed by updated vertex set $V_i$ of $R_i$
            \State Let $x = z$
            \State Let $P = P \cup \{y\}$
            \State Let $t = t + 1$
        \EndIf
    \EndWhile
    \State \textbf{return} P
    \end{algorithmic}
\end{algorithm}

Algorithm \ref{alg:contractionPathGeneration} performs the first phase of MCA in \cite{MalinaoJuayongScienggj2024} with some modification. Instead of doing the contraction until the sink, this algorithm \ref{alg:contractionPathGeneration} performs the contraction until there is no vertex that can be contracted, leaving the unreachable vertices as not part of the merged vertices. This process involves checking the constraints coming from vertex $x$ to $y$ and verifying whether those set of constraints is a superset of all the constraints of arcs coming towards $y$. If so, then $x$ and $y$ can be contracted. With this procedure, $Points of Delay$ ($POD$)\cite{Malinao2017} vertices will not be merged, at least if there is no looping arc towards those points which can resolve the constraints making them reachable at some time step $t$. $POD$s are vertices that cannot be visited immediately due to the unresolved constraints of its incoming arcs.


\subsubsection*{Verification of the Existence of Escape Conctraction Paths}
After the identification of deadlocks in $R$, in order to verify whether or not it is deadlock resolving as per its definition, each deadlock should have an escape contraction path \cite{Ramirez}. Additionally, the parents involved in this verification are only the reachable ones.

Then, the proposed algorithm will go through each parent of each deadlock to verify the existence of an escape contraction path. 


\begin{algorithm}[H]
\caption{\textbf{Matrix-Based Finding Escape Contraction Path Algorithm}}
\label{alg: findEscapeConractionPath}
\textbf{Given}: RDLT $R_i$; Preprocessing


\textbf{Input}: Parent Vertex $w$, Sink $f'$, Contraction Path $P$, Adjacency Matrix of $R_i$ $RV_{adj}$

\textbf{Output}: Boolean, $True$, otherwise $False$  

\begin{algorithmic}[1]
    \State Initialize $Visited[0...n-1] \gets false$ \Comment for each vertex in $V_i$
    \State Initialize $Q \gets$ empty queue
    \State $Q.queue(w)$
    \State $Visited[w] \gets true$
    \While{$Q \neq \emptyset$}
        \State $current \gets Q.dequeue()$
        \State Let $k$ be the index of $current$ vertex in $V_i$
        \If{$current = f'$}
            \State \textbf{return }$True$
            \ForAll{vertex $x$ in $P$}  
                \State  Let $l$ be the index of $x$ in $V_i$
                \If {$RV_{adj}[k][l] = 1 \land Visted[l]= false $}
                    \State $Q.enqueue(x)$
                    \State $Visited[l] \gets true$
                \EndIf
            \EndFor
        \EndIf
    \EndWhile
    \State \textbf{return }$False$
    
\end{algorithmic}
\end{algorithm}

Algorithm \ref{alg: findEscapeConractionPath} goes through the each parents of a deadlock point and finds an escape contraction path with it. A breadth-first search is used in this algorithm, and the escape contraction path is a path induced from the contaction path from the algorithm \ref{alg:contractionPathGeneration}. It means that the escape contraction path is a path from the parent to the sink, and at the same time reachable, which is implied if there is a path that can be induced from the contraction path $P$. Algorithm \ref{alg: findEscapeConractionPath} outputs a boolean, signifying whether or not there is an escape contraction path for the input deadlock.

Now that the deadlocks and escape contraction paths can be identified through the use of matrices and matrix operations, deadlock-resolving verification can now be implemented. 

\begin{algorithm}[H]
    % \label{alg:MB-deadlock-resolving}
    \caption{Matrix-based Deadlock-resolving Verification Algorithm}

    \textbf{Input:} Expanded Vertex Simplification $R_i$ of RDLT $R$, Contraction Path Algorithm Generation \ref{alg:contractionPathGeneration}, Find Escape Contraction Path Algorithm \ref{alg: findEscapeConractionPath}

    \textbf{Output:} Boolean; $True$, otherwise $False$

    \textbf{Matrices:} Adjacency Matrix $RV_{adj}$, Constraint matirx $RV_C$

    \begin{algorithmic}[1]
        \State $P \gets Contraction Path Generation Algorithm (R_i)$ \ref{alg:contractionPathGeneration}
        \Statex \textbf{Deadlock Identification:} Vertices not in P, but are adjacent to P, and are PODs
        \State $dV \gets \emptyset$ \Comment{Initialize deadlock set}
        \State $RV_{adj} \gets RV^0_{adj}$
        \State $RV_{C} \gets RV^0_{C}$
        \State $unreachable\_vertices \gets V_i / P$ 
        \ForAll{vertex $x \in unreachable\_vertices$}   
            \State Let $l$ be the index of vertex $x$ in $V_i$
            \State $constraints\_stack \gets \emptyset$
            \State $constraints\_counter\gets 0$

            \ForAll{row $k$ in $RV_C$} 
                \State $c \gets RV_C[k][l]$
                \If {$c \neq \emptyset \land c \notin \{\epsilon\} \cup constraints\_stack$}
                    \State $constraints\_stack \gets constraints\_stack \cup \{c\}$
                    \State $constraint\_counter \gets constraint\_counter + 1$
                \EndIf
            \EndFor
            \If{$constraints\_counter \geq 2$}
                \State $dV \gets x_l \cup dV$
            \EndIf
        \EndFor
        \Statex \textbf{Enumeration of Parents of Deadlocks and Finding Escape Contraction Pahts}
        \ForAll {vertex $x \in dV$}    
            \State $parents(x) \gets \emptyset$
            \State Let $l$ be the index of vertex $x$ in $V_i$
            \ForAll {row $k$ in $RV_{adj}$}
                \If {$RV_{adj}[k][l] = 1 \land x_k \in P$} \Comment{The parent vertex needs to be reachable, hence included in the contraction path}
                    \State $parent(x) \gets x_k \cup parent(x)$
                \EndIf
            \EndFor
            \ForAll {parent vertex $w \in parent(x)$}
                % what do I need to include in here? 
                \State $escapePathExists \gets findEscapePath(w, f', P, RV_{})$ \ref{alg: findEscapeConractionPath}
                \If{$escapePathExists = false$}
                    \State \textbf{return} $False$
                \EndIf
            \EndFor
        \EndFor
        \State \textbf{return} $True$
    \end{algorithmic}
\end{algorithm}

This matrix-based verification of whether or not an RDLT $R$ is deadlock-resolving or not involves the generation of contraction path, deadlock identification, parent enumeration for each deadlock, and verification of the existence of escape contraction path for each parent vertex. On deadlock identification, these are the vertices not part of the contraction path $P$ and are not $POD$s. Using the matrices $RV_{adj}$ and $RV_{C}$ which is the adjacency matrix before any contraction deadlocks can be identified, which are the immediate $POD$s adjacent from the merged vertex through algorithm \ref{alg:contractionPathGeneration}.

On the identification of the parents for each deadlocks, martix $RV_{adj}$ is used as well as the contraction path $P$. Parent vertices used in finding contraction paths should be at least reachable, hence $P$ is used. 

To verify the existence of escape contraction path, there should be an existing path from the parent vertex $w$ to the sink vertex $f'$. Using \ref{alg: findEscapeConractionPath}, it will return a boolean verifying whether or not a path can be induced from the contraction path $P$ from the parent vertex $w$ to the sink $f'$. 


\textbf{Formal algorithm for Weakened JOIN Safeness and Deadlock Tolerance to be added ASAP}
\textbf{Matrix-based Easy Soundness Verification to be added ASAP}
\textbf{Revisions and Edits from Ch 1 to 2 and Results and Discussion to be added ASAP}

% \subsubsection*{Enumeration of Parents of Each Deadlocks}
% After the identification of deadlocks in $R$, in order to verify whether or not it is deadlock resolving as per its definition, each deadlock should have an escape contraction path \cite{Ramirez}. Additionally, the parents involved in this verification are only the reachable ones.


\begin{comment}
    What should I include in this part?
    Include the contraction algorithm by Joenne. And the subsequent matrices that I should be defining. 
    - Adjacency Matrix
    - Constraints Matrix
    - Modified Contraction Paths Generation Algorithm 
    TODO: Include in Related Literature necessary defintions in Weak and Easy Sound Verification such as Deadlock-resolving and escape contraction path. 
    TODO: Include definition of PODs
    TODO: Update the definitions in Ramirez RRL
    
\end{comment}

