% ------------------------------------------------------------------------
% MS Thesis of Nathaniel Enrique Eulin 
% Date Defended: May 1, 2025 
% ------------------------------------------------------------------------
\documentclass[letterpaper,12pt]{report}
\usepackage[centertags]{amsmath}
\usepackage{amsfonts}
\usepackage{amssymb}
\usepackage{amsthm}
\usepackage{newlfont}
\usepackage{epsfig}
\usepackage{UPCSTHESIS} %This loads the UP Computer Science Thesis Package
\usepackage{UPCSINC1}
\usepackage[active]{LTX}
\usepackage{subfigure}
\usepackage{listings}
\usepackage{array}
\usepackage{graphicx}
%\usepackage{float}
\usepackage{url}
\usepackage{lscape}

\usepackage{fancyhdr}

\fancyhf{} % clear all header and footer fields
\fancyhead[R]{\thepage}

\renewcommand{\headrulewidth}{0pt}
\renewcommand{\footrulewidth}{0pt}

% Redefining plain style which is automatically applied to chapters (including Bibliography)

\fancypagestyle{plain}{%
\fancyhf{} % clear all header and footer fields
\fancyhead[R]{\thepage}
\renewcommand{\headrulewidth}{0pt}
\renewcommand{\footrulewidth}{0pt}}

\pagestyle{fancy} 

%% Define a new 'leo' style for the package that will use a smaller font.
\makeatletter
\def\url@leostyle{%
  \@ifundefined{selectfont}{\def\UrlFont{\sf}}{\def\UrlFont{\small\ttfamily}}}
\makeatother
%% Now actually use the newly defined style.
\urlstyle{leo}


\hfuzz2pt
\newlength{\defbaselineskip}
\setlength{\defbaselineskip}{\baselineskip}
\newcommand{\setlinespacing}[1]%
           {\setlength{\baselineskip}{#1 \defbaselineskip}}
\newcommand{\doublespacing}{\setlength{\baselineskip}%
                           {2.0 \defbaselineskip}}
\newcommand{\singlespacing}{\setlength{\baselineskip}{\defbaselineskip}}
% MATH -------------------------------------------------------------------
\newcommand{\A}{{\cal A}}
\newcommand{\h}{{\cal H}}
\newcommand{\s}{{\cal S}}
\newcommand{\W}{{\cal W}}
\newcommand{\BH}{\mathbf B(\cal H)}
\newcommand{\KH}{\cal  K(\cal H)}
\newcommand{\Real}{\mathbb R}
\newcommand{\Complex}{\mathbb C}
\newcommand{\Field}{\mathbb F}
\newcommand{\RPlus}{[0,\infty)}
\newcommand{\norm}[1]{\left\Vert#1\right\Vert}
\newcommand{\essnorm}[1]{\norm{#1}_{\text{\rm\normalshape ess}}}
\newcommand{\abs}[1]{\left\vert#1\right\vert}
\newcommand{\set}[1]{\left\{#1\right\}}
\newcommand{\seq}[1]{\left<#1\right>}
\newcommand{\eps}{\varepsilon}
\newcommand{\To}{\longrightarrow}
\newcommand{\RE}{\operatorname{Re}}
\newcommand{\IM}{\operatorname{Im}}
\newcommand{\Poly}{{\cal{P}}(E)}
\newcommand{\EssD}{{\cal{D}}}
% THEOREMS ---------------------------------------------------------------
\theoremstyle{plain}
\newtheorem{thm}{Theorem}[section]
\newtheorem{cor}[thm]{Corollary}
\newtheorem{lem}[thm]{Lemma}
\newtheorem{prop}[thm]{Proposition}
\theoremstyle{definition}
\newtheorem{defn}{Definition}[section]
\theoremstyle{remark}
\newtheorem{rem}{Remark}[section]
\numberwithin{equation}{section}
\renewcommand{\theequation}{\thesection.\arabic{equation}}
\setlength{\tclineskip}{1.05\baselineskip}

\begin{document}

%\nobib  			%toggles bibliography control
%\draft				%toggles draft printing
%\nofront			%toggles nofront printing

\permissionfalse

%\nolistoftables	%toggles table control
%\nolistoffigures	toggles figures

% ------------------------------------------------------------------------

\title{(Matrix Representation for the Verification of Weak and Easy Soundness in Robustness Diagrams with Loops and Time Controls)}

\bs
\author{(Nathaniel Enrique Catalan Eulin)}
\degreeinitial{B.S.C.S.}
\major{Computer Science}
\adviser{Dr. techn. Jasmine A. Malinao}
\firstreader{(TBD)}
\secondreader{(TBD)}
\dean{Dr. Patricia B. Arinto}
\submityear{(May) (2025)}
\studentid{(2021-08363)}

% ------------------------------------------------------------------------
{
%\WinEdt{?0000} % Don't bother with over/under-full boxes
%\beforepreface
%\WinEdt{?1111} % Process All Errors from Here on
\forgeterr{?0000} % Don't bother with over/under-full boxes
\beforepreface
\forgeterr{?1111} % Process All Errors from Here on
}

{
\typeout{Acknowledgements}
% \include{acknowl}
}
% ------------------------------------------------------------------------
{
\typeout{Abstract}
% \include{abstract}
}
\tableofcontentspage
% ------------------------------------------------------------------------
%\setcounter{page}{1}
%\tableofcontents
% ------------------------------------------------------------------------

% ------------------------------------------------------------------------
\afterpreface
\def\baselinestretch{1}
\setlinespacing{1.66}
% ------------------------------------------------------------------------
{
%\typeout{Introduction}
%\include{introd}
}
% ------------------------------------------------------------------------
\setlinespacing{1.66}
% ------------------------------------------------------------------------

\include{chapter1} %iintroduction
\include{chapter2} %RRL
\chapter{Methodology}

\begin{comment}
    Outline of the Methodology:
    - Introduction
    - Verification of Deadlock-Resolving RDLT
        - Contraction Path Generation (for identificaiton of deadlocks)
        - identification of escape contraction paths
        - Matrix-based algorithm for verification of Deadlock-Resolving RDLT
    - Loop-Safeness and Safeness Verification (Matrix-based)
    - Weakened JOIN-Safeness Verification (Matrix-based)
    - Weak Soundness of RDLT Verification (Matrix-based)
    - Easy Soundness of RDLT Verification (Matrix-based)

\end{comment}

This section introduces a matrix-based implementation of the verfication of weak and easy soundness of RDLT. The general algorithm for these verification is based on the formalizations of \cite{Ramirez2024} in terms of structural verifications of the said soundness properties. 

Firstly, on weak soundness, \cite{Ramirez2024} proposed WRSVA in which Deadlock-Tolerance is verified in order to verify weather or not an RDLT is weak sound. As explained in the related literatures, deadlock-tolerance refers to an RDLT being deadlock-resolving, having all NCAs loop-safe, CAs safe, and having all its split-join pairs as weakened JOIN-safe. All of these properties are verified through the use of matrices and their corresponding operations. The proofs and test-cases are also outlined for each verification of these mentioned properties. 

Similarly, easy soundness verification as formalized by \cite{Ramirez2024} is also implemented using matrix representation and operations. Its property of having an option to complete is verified by finding a contraction path from the source of R to the sink. Corresponding proofs and test-cases are also included to this proposed method. 

\subsection{Verification of Deadlock-Resolving RDLT}
% I think I should include a reference of the theorems from Ramirez' regarding his definitions. 
Theorem 3.4.3 in \cite{Ramirez2024} explains that an RDLT $R$ is of weak soundness if and only if both level-1 and level-2 vertex simplifications of $R$ are deadlock-tolerance. The paper defines deadlock-tolerance as an RDLT where every $NCA$ are loop-safe, every $CA$ are safe, $R$ is weakened JOIN-safe, and is deadlock-resolving.

For the verification of whether $R$ is deadlock-resolving, the matrix representation of which includes the enumeration of relevant deadlocks in $R$, and the identification of the existence of escape contraction paths from the parent of each deadlocks. The existence of an escape contraction paths is what ensures proper termination albeit the existence of deadlocks within an RDLT, removing the liveness property. Therefore, such RDLT is of weak soundness.

\subsubsection*{Identification of Deadlocks}
On the identification deadlocks, a method is proposed which involves the contraction \cite{Malinao2017} of the expanded vertex simplified RDLT $R_i$. This contraction algorithm as proposed by Malinao in here dissertation checks the c-verifiability of the RDLT \cite{Malinao2017}. Two adjacent vertices are contractable or can be merged when the arc currently being checked has contraint/s which is a subset of all the contraints coming towards the other vertex the algorithm is trying to contract. The final contraction shows the contraction path, which is a path of vertcies, that are reachable through an activity, hence through activity extraction, as well. That said, no unreachable vertices will be included in the contracted vertices, hence deadlocks, as defined in \cite{Ramirez2024}, can be identified.

This paper implements a matrix-based contraction algorithm adopted from the modified contraction algorithm (MCA) introduced in \cite{malinao2024model}. The following sections define the necessary matrices and algorithms to achieve a research objective. 

% Matrices 

\begin{defn} \textbf{Adjacency Matrix $RV_{adj}$} \\
Let $R_i = \bigl(V_i, E_i,\Sigma_i, C_i, L_i, M_i\bigr)$ be an expanded vertex simplification of an RDLT $R$. The $n \times n$ matrix of a directional graph, where $n = |V_i|$ and the rows and columns correspond to the vertex set $V_i = \{\,v_1, v_2, \ldots, v_n\}$ at time step $t$, is defined as

\[
  RV_{adj}^{t} \;=\;
  \begin{pmatrix}
    RV_{\mathrm{adj}}(v_1,v_1) & RV_{\mathrm{adj}}(v_1,v_2) & \cdots & RV_{\mathrm{adj}}(v_1,v_n) \\[6pt]
    RV_{\mathrm{adj}}(v_2,v_1) & RV_{\mathrm{adj}}(v_2,v_2) & \cdots & RV_{\mathrm{adj}}(v_2,v_n) \\[6pt]
    \vdots                    & \vdots                    & \ddots & \vdots                    \\[6pt]
    RV_{\mathrm{adj}}(v_n,v_1) & RV_{\mathrm{adj}}(v_n,v_2) & \cdots & RV_{\mathrm{adj}}(v_n,v_n)
  \end{pmatrix}.
\]

Each entry $RV_{adj}(x,y)$ represents the number of distinct arcs from $x$ to $y$, given by
\[
  RV_{adj}(x,y) \;=\;
  \begin{cases}
    RV_{adj}(x,y) \in \mathbb{N}, & \text{if } (x,y)\,\in E_i \text{ in } R_i, \\[5pt]
    0,     & \text{otherwise}.
  \end{cases}
\]
\end{defn}

\begin{defn} \textbf{Constraints Matrix $RV_C$} \\
We similarly define the matrix $RV_c$. The $n \times n$ matrix of a directional graph, where $n = |V_i|$ and the rows and columns correspond to the vertex set $V_i = \{\,v_1, v_2, \ldots, v_n\}$ at time step $t$, is defined as
\[
   RV_{C}^{t} \;=\;
  \begin{pmatrix}
    RV_c(v_1,v_1) & RV_c(v_1,v_2) & \cdots & RV_c(v_1,v_n) \\[6pt]
    RV_c(v_2,v_1) & RV_c(v_2,v_2) & \cdots & RV_c(v_2,v_n) \\[6pt]
    \vdots        & \vdots        & \ddots & \vdots        \\[6pt]
    RV_c(v_n,v_1) & RV_c(v_n,v_2) & \cdots & RV_c(v_n,v_n)
  \end{pmatrix}.
\]
Each entry $RV_{C}^{t}(x,y)$ represents the corresponding C-attribute of the arcs from $x$ to $y$ in the Adjacency Matrix, given by
\[
  RV_{C}^{t}(x,y) \;=\;
  \begin{cases}
    C(x,y)\in \Sigma \cup \{\epsilon\}, & \text{if } (x,y)\in E_i \text{ in } R_i,\\[5pt]
    \varnothing,                                  & \text{otherwise}.
  \end{cases}
\]
\end{defn}

\newpage
 
\begin{algorithm}[H]
\caption{\textbf{Matrix-based Contraction Path Generation}}
\label{alg:contractionPathGeneration}
\textbf{Given} RDLT R; Pre-processing Steps:

\textbf{Input}: Expanded vertex simplification $R_i$ of RDLT R

\textbf{Output}: Contraction Path P

\textbf{Matrices}: $RV^{t}_{\text{adj}}$ and $RV^{t}_{\text{C}}$

    \begin{algorithmic}[1]

    \State Initialize C-Attribute Matrix $RV^{0}_{\text{C}}$ of $R_i$
    \State Let $s' \in V_i$ be the source and $f' \in V_i$ be the sink
    \State Let $x = s'$
    \State Initialize $P = \{x\}$
    \State Let $t = 1$

    % \While{$P$ does not contain $f'$}
    % \While{\exists\, x \in X}
    % \While{\exists $y$ \in $V_i$}
    \While{$\exists y \in V_i \mid RV^{t-1}_{\text{adj}}(x,y) \geq 1$}

        \State $\mathcal{Y} \leftarrow \{ y \in V_i \mid RV^{t-1}_{\text{adj}}(x,y) \geq 1 \}$
        \State Select any $y \in \mathcal{Y}$
        \State Let $\text{LHS} = RV^{t-1}_{C}(x,y) \cup \{\epsilon\}$
        \State $\mathcal{U} \leftarrow \{ u \in V_i \mid u \neq x \wedge (RV^{t-1}_{\text{adj}}(u,y) \geq 1) \}$
        \State Let $\text{RHS} = \bigcup_{u \in \mathcal{U}} RV^{t-1}_{C}(u,y)$
        \If {$\text{LHS} \supseteq \text{RHS}$}
            \State Update $RV^{t-1}_{C}(u,y) = \epsilon, \forall (u,y) \in E_i, u \neq x$
            \ForAll {$u \in \mathcal{U}$}
                \State $RV^{t-1}_{C}(u,y) = \epsilon$
            \EndFor
            \State Let $z = x \land y$
            \State Let $z = xy$ = Matrix Addition of rows (columns) $x$ and $y$ in $RV^{t-1}_{\text{adj}}$
            \ForAll {$w \in V_i$}
                \State RowMerge\_Adj: $RV^{t}_{\text{adj}}(z,w) = RV^{t-1}_{\text{adj}}(x,w) + RV^{t-1}_{\text{adj}}(y,w)$
                \State ColMerge\_Adj: $RV^{t}_{\text{adj}}(w,z) = RV^{t-1}_{\text{adj}}(w,x) + RV^{t-1}_{\text{adj}}(w,y)$
            \EndFor
            \State Let $z = xy$ = Element-wise Set Union of rows (columns) $x$ and $y$ in $RV^{t-1}_{C}$
            \ForAll {$w \in V_i$}
                \State RowMerge\_C: $RV^{t}_{C}(z,w) = RV^{t-1}_{C}(x,w) \cup RV^{t-1}_{C}(y,w)$
                \State ColMerge\_C: $RV^{t}_{C}(w,z) = RV^{t-1}_{C}(w,x) \cup RV^{t-1}_{C}(w,y)$
            \EndFor
            \State $V_i = (V_i \setminus \{x,y\}) \cup \{z\}$
            \State Create $RV^{t}_{\text{adj}}$ as an $m \times m$ matrix where $m = n - t$ and as the submatrix of $RV^{t-1}_{\text{adj}}$ with rows and columns indexed by updated vertex set $V_i$ of $R_i$
            \State Create $RV^{t}_{C}$ as an $m \times m$ matrix where $m = n - t$ and as the submatrix of $RV^{t-1}_{C}$ with rows and columns indexed by updated vertex set $V_i$ of $R_i$
            \State Let $x = z$
            \State Let $P = P \cup \{y\}$
            \State Let $t = t + 1$
        \EndIf
    \EndWhile
    \State \textbf{return} P
    \end{algorithmic}
\end{algorithm}

Algorithm \ref{alg:contractionPathGeneration} performs the first phase of MCA in \cite{MalinaoJuayongScienggj2024} with some modification. Instead of doing the contraction until the sink, this algorithm \ref{alg:contractionPathGeneration} performs the contraction until there is no vertex that can be contracted, leaving the unreachable vertices as not part of the merged vertices. This process involves checking the constraints coming from vertex $x$ to $y$ and verifying whether those set of constraints is a superset of all the constraints of arcs coming towards $y$. If so, then $x$ and $y$ can be contracted. With this procedure, $Points of Delay$ ($POD$)\cite{Malinao2017} vertices will not be merged, at least if there is no looping arc towards those points which can resolve the constraints making them reachable at some time step $t$. $POD$s are vertices that cannot be visited immediately due to the unresolved constraints of its incoming arcs.


\subsubsection*{Verification of the Existence of Escape Conctraction Paths}
After the identification of deadlocks in $R$, in order to verify whether or not it is deadlock resolving as per its definition, each deadlock should have an escape contraction path \cite{Ramirez}. Additionally, the parents involved in this verification are only the reachable ones.

Then, the proposed algorithm will go through each parent of each deadlock to verify the existence of an escape contraction path. 


\begin{algorithm}[H]
\caption{\textbf{Matrix-Based Finding Escape Contraction Path Algorithm}}
\label{alg: findEscapeConractionPath}
\textbf{Given}: RDLT $R_i$; Preprocessing


\textbf{Input}: Parent Vertex $w$, Sink $f'$, Contraction Path $P$, Adjacency Matrix of $R_i$ $RV_{adj}$

\textbf{Output}: Boolean, $True$, otherwise $False$  

\begin{algorithmic}[1]
    \State Initialize $Visited[0...n-1] \gets false$ \Comment for each vertex in $V_i$
    \State Initialize $Q \gets$ empty queue
    \State $Q.queue(w)$
    \State $Visited[w] \gets true$
    \While{$Q \neq \emptyset$}
        \State $current \gets Q.dequeue()$
        \State Let $k$ be the index of $current$ vertex in $V_i$
        \If{$current = f'$}
            \State \textbf{return }$True$
            \ForAll{vertex $x$ in $P$}  
                \State  Let $l$ be the index of $x$ in $V_i$
                \If {$RV_{adj}[k][l] = 1 \land Visted[l]= false $}
                    \State $Q.enqueue(x)$
                    \State $Visited[l] \gets true$
                \EndIf
            \EndFor
        \EndIf
    \EndWhile
    \State \textbf{return }$False$
    
\end{algorithmic}
\end{algorithm}

Algorithm \ref{alg: findEscapeConractionPath} goes through the each parents of a deadlock point and finds an escape contraction path with it. A breadth-first search is used in this algorithm, and the escape contraction path is a path induced from the contaction path from the algorithm \ref{alg:contractionPathGeneration}. It means that the escape contraction path is a path from the parent to the sink, and at the same time reachable, which is implied if there is a path that can be induced from the contraction path $P$. Algorithm \ref{alg: findEscapeConractionPath} outputs a boolean, signifying whether or not there is an escape contraction path for the input deadlock.

Now that the deadlocks and escape contraction paths can be identified through the use of matrices and matrix operations, deadlock-resolving verification can now be implemented. 

\begin{algorithm}[H]
    % \label{alg:MB-deadlock-resolving}
    \caption{Matrix-based Deadlock-resolving Verification Algorithm}

    \textbf{Input:} Expanded Vertex Simplification $R_i$ of RDLT $R$, Contraction Path Algorithm Generation \ref{alg:contractionPathGeneration}, Find Escape Contraction Path Algorithm \ref{alg: findEscapeConractionPath}

    \textbf{Output:} Boolean; $True$, otherwise $False$

    \textbf{Matrices:} Adjacency Matrix $RV_{adj}$, Constraint matirx $RV_C$

    \begin{algorithmic}[1]
        \State $P \gets Contraction Path Generation Algorithm (R_i)$ \ref{alg:contractionPathGeneration}
        \Statex \textbf{Deadlock Identification:} Vertices not in P, but are adjacent to P, and are PODs
        \State $dV \gets \emptyset$ \Comment{Initialize deadlock set}
        \State $RV_{adj} \gets RV^0_{adj}$
        \State $RV_{C} \gets RV^0_{C}$
        \State $unreachable\_vertices \gets V_i / P$ 
        \ForAll{vertex $x \in unreachable\_vertices$}   
            \State Let $l$ be the index of vertex $x$ in $V_i$
            \State $constraints\_stack \gets \emptyset$
            \State $constraints\_counter\gets 0$

            \ForAll{row $k$ in $RV_C$} 
                \State $c \gets RV_C[k][l]$
                \If {$c \neq \emptyset \land c \notin \{\epsilon\} \cup constraints\_stack$}
                    \State $constraints\_stack \gets constraints\_stack \cup \{c\}$
                    \State $constraint\_counter \gets constraint\_counter + 1$
                \EndIf
            \EndFor
            \If{$constraints\_counter \geq 2$}
                \State $dV \gets x_l \cup dV$
            \EndIf
        \EndFor
        \Statex \textbf{Enumeration of Parents of Deadlocks and Finding Escape Contraction Pahts}
        \ForAll {vertex $x \in dV$}    
            \State $parents(x) \gets \emptyset$
            \State Let $l$ be the index of vertex $x$ in $V_i$
            \ForAll {row $k$ in $RV_{adj}$}
                \If {$RV_{adj}[k][l] = 1 \land x_k \in P$} \Comment{The parent vertex needs to be reachable, hence included in the contraction path}
                    \State $parent(x) \gets x_k \cup parent(x)$
                \EndIf
            \EndFor
            \ForAll {parent vertex $w \in parent(x)$}
                % what do I need to include in here? 
                \State $escapePathExists \gets findEscapePath(w, f', P, RV_{})$ \ref{alg: findEscapeConractionPath}
                \If{$escapePathExists = false$}
                    \State \textbf{return} $False$
                \EndIf
            \EndFor
        \EndFor
        \State \textbf{return} $True$
    \end{algorithmic}
\end{algorithm}

This matrix-based verification of whether or not an RDLT $R$ is deadlock-resolving or not involves the generation of contraction path, deadlock identification, parent enumeration for each deadlock, and verification of the existence of escape contraction path for each parent vertex. On deadlock identification, these are the vertices not part of the contraction path $P$ and are not $POD$s. Using the matrices $RV_{adj}$ and $RV_{C}$ which is the adjacency matrix before any contraction deadlocks can be identified, which are the immediate $POD$s adjacent from the merged vertex through algorithm \ref{alg:contractionPathGeneration}.

On the identification of the parents for each deadlocks, martix $RV_{adj}$ is used as well as the contraction path $P$. Parent vertices used in finding contraction paths should be at least reachable, hence $P$ is used. 

To verify the existence of escape contraction path, there should be an existing path from the parent vertex $w$ to the sink vertex $f'$. Using \ref{alg: findEscapeConractionPath}, it will return a boolean verifying whether or not a path can be induced from the contraction path $P$ from the parent vertex $w$ to the sink $f'$. 


\textbf{Formal algorithm for Weakened JOIN Safeness and Deadlock Tolerance to be added ASAP}
\textbf{Matrix-based Easy Soundness Verification to be added ASAP}
\textbf{Revisions and Edits from Ch 1 to 2 and Results and Discussion to be added ASAP}

% \subsubsection*{Enumeration of Parents of Each Deadlocks}
% After the identification of deadlocks in $R$, in order to verify whether or not it is deadlock resolving as per its definition, each deadlock should have an escape contraction path \cite{Ramirez}. Additionally, the parents involved in this verification are only the reachable ones.


\begin{comment}
    What should I include in this part?
    Include the contraction algorithm by Joenne. And the subsequent matrices that I should be defining. 
    - Adjacency Matrix
    - Constraints Matrix
    - Modified Contraction Paths Generation Algorithm 
    TODO: Include in Related Literature necessary defintions in Weak and Easy Sound Verification such as Deadlock-resolving and escape contraction path. 
    TODO: Include definition of PODs
    TODO: Update the definitions in Ramirez RRL
    
\end{comment}

 %methodology
% \include{chapter4} %results and discussion
% \include{chapter5} %conclusions and recommendations
% \appendix
\section*{Appendix A: Activity Extraction Algorithm}
\begin{algorithm}[H]
    \caption{Activity Extraction Algorithm ${\mathcal A}$ \cite{Malinao2017, Sulla2023}}
    \label{ActivityExtraction}
    \begin{algorithmic}
        \State \textbf{Input:} $ R $, $ s \in V $, $ f \in V $
        \State \textbf{Output:} vertices $ S $ of $ R $, $ \emptyset $ otherwise
    \end{algorithmic}
    \begin{algorithmic}[1]
        \State Initialize $ S $
        \For{every $ (x,y) $}
            \State Initialize $ T((x,y)) $ such that $ T((x,y)) $ $ = $ $ (t_1, ..., t_n) $ where $ n $ $ = $ $ L((x,y)) $ and $ t_i $ $ \epsilon $ $ \mathbb{N} $ is the time a check or traversal is done on $ (x,y) $ by $ A $.
        \EndFor
        \State Let $ x $ $ = $ $ s $.
        \While{$ x $ $ \neq $ $ f $}
            \State Select $ (x,y) $ $ \epsilon $ $ E $ where $ L((x,y)) $ has not been reached.
            \If{$ \exists (u,x) $ $ \epsilon $ $ E $ $ = $ $ max(T((u,x))) $}
                \State Assign $ maxV $ $ + $ $ 1 $ to the leftmost zero of $ T((x,y)) $. 
            \Else
                \State $ maxV $ $ = $ $ 0 $.
            \EndIf
            \State Determine whether $ (x,y) $ is an unconstrained arc or not.
            \If{$ (x,y) $ is unconstrained}
                \State Traverse $ (x,y) $ .
                \State Assign $ MAX $ $ + $ $ 1 $ to $ T((x,y)) $ where $ MAX $ is the maximum value from all $ T((v',y)) $ $ \forall $ $ v' $ $ \epsilon $ $ V $ where $ (v',y),(v,y) $, and $ (x,y) $ are type-alike.
                \For{every $ (v,y) $ $ \epsilon $ $ E $ that is type-alike with $ (x,y) $}
                    \State Assign $ MAX $ $ + $ $ 1 $ to every $ T((x,y)) $.
                    \If{$ C((v,y)) $ $ \epsilon $ $ \Sigma $}
                        \State The last value in $ T((x,y)) $ where the last check was done on $ (v,y) $ is updated.
                    \ElsIf{$ v $ is either type 'b' or 'e' and $ y $ is type 'c'}
                        \State The first value in $ T((x,y)) $ is updated.
                    \EndIf
                \EndFor
            \ElsIf{$ (x,y) $ is not unconstrained and not other $ (x,y') $ $ \epsilon $ $ E $ where $ y' $ $ \epsilon $ $ V $ can be selected }
                \State Backtrack to $ a $ $ \epsilon $ $ V $ where $ (a,x) $ $ \epsilon $ $ E $ and $ a $ was previously visited by the algorithm to reach $ x $.
            \EndIf
        \EndWhile
        \If{activity extraction fails}
            \State \Return $ \emptyset; $
        \Else
            \State \Return $ S $
        \EndIf
    \end{algorithmic}
\end{algorithm}

\section*{Appendix B: Check Subroutine}
\begin{algorithm} 
    \begin{algorithmic}
        \State \textbf{Input:} $x \in V$
        \State \textbf{Output:} $y \in V$ if the arc attribute $L((x,y))$ allows that at least one traversal on the arc, $\emptyset$ otherwise
    \end{algorithmic}
    \begin{algorithmic}[1]
        \State $y \gets w$ where $w \in V$, $(x,w) \in E$ and for $1 \leq i \leq |L(x,w)|$ such that either $CTInd_{(x,y)}[i-1] = 2$ and $CTInd_{(x,y)}[i] = 0$, or $CTInd_{(x,y)}[i=1] = 0$\;
        \If {$y \neq \emptyset$} 
            \If {$\exists(u,x) \in E, u \in V$}
                \State $t_i \in T((x,y)): t_i \gets maxV + 1$ where $maxV = \displaystyle\max_{\forall u, (u,x) \in E} \{{\displaystyle\max_{k=1, 2, \ldots, L((u,x)),  t_k \in T((u,x))}{\{t_k\}}}\} + 1$\;
            \Else
                \State $t_i \in T((x,y)): t_i \gets 1$\;
            \EndIf
            \State return $y$;
        \Else
            \State return $0$;
        \EndIf
    \end{algorithmic}
   \caption{$Check(x)$: Determines if there exists some $y \in V$ where $(x,y) \in E$ and its attribute $L((x,y))$ allows that at least one traversal on the arc. If $y$ exists, the algorithm updates $T((x,y))$ and returns $y$, otherwise it returns $\emptyset$.}
   \label{alg:Check}
\end{algorithm}
\section*{ Expanded Vertex Simplification Algorithm (EVSA)}
\begin{algorithm}
    \label{EVSA}
    \underline{\textbf{Expanded Vertex Simplification Algorithm(EVSA):}}\\
    \textbf{Inputs:} Let $R = (V, E, T, M)$ be an connected RDLT with an RBS, i.e. $\exists v \in V$, where $M(v) = 1$. Let $R_1 = (V_1, E_1)$ and $R_2 = (V_2, E_2)$ be the vertex simplifications of $R$. (The $C$-attributes of the arcs of $R$, $R_1$, and $R_2$ are denoted as $C$, $C_1$, $C_2$, respectively, while their $L$-attributes are $L$, $L_1$, and $L_2$, respectively.) \\

    \noindent
    \textbf{Outputs:} \textbf{Level 1 and Level 2 expanded vertex simplification} $R'_1 = (V'_1, E'_1, T'_1)$ and $R'_2 = (V'_2, E'_2, T'_2)$ of $R$, respectively . (The $C$-attributes of the arcs of $R'_1$ and $R'_2$ are $C'_1$ and $C'_2$, respectively, while their $L$-attributes are $L'_1$ and $L'_2$, respectively.)

    \begin{enumerate}
        \item For each $v \in V_1(V_2)$(or $V$ of $R$), set $v' \in V'_1(V'_2)$ to be its corresponding vertex, and if $(u,v) \in E_1(E_2)$, then its there is a corresponding arc $(u',v') \in E'_1(E'_2)$, where the $C$-values $C'_1(u',v')(C_2(u',v'))$ of $R'_1(R'_2)$ is set to $C(u,v)$ of $R_1(R_2)$, otherwise, none.  \\

        This step copies the vertices and arcs, inclusive of the $C$-values of $R_1$ and $R_2$ to their corresponding expanded versions $R'_1$ and $R'_2$, respectively. 

        \item For each $(u,v) \in E$, where $(u,v)$ is not in an RBS of $R$, and its corresponding arc $(u',v') \in V'_1$ of $L_1$, let $L'_1(u',v') = L(u,v)$. \\

        This step copies the $L$-values of $R$ for each of its arcs to the $L$-values of $R'_1$ whenever such arc does not belong to an RBS of $R$. 
        
        \item For each abstract arc $(u',v') \in E'_1$ of $R_1$, where $(u',v')$ represents the path \mbox{$p = x_1 x_2 \ldots x_n$} in $R$, i.e. $x_1 \in V$ and $x_n \in V$ correspond to $u' \in E'_1$ and  $v' \in E'_1$, respectively, set $L'_1(u',v') = \min\limits_{i = 1, \ldots, n-1} \{eRU(x_i, x_{i+1})\} + 1$. \\
        
        This step sets the $L$-value  of each abstract arc $(u',v')$ of $R'_1$ to reflect the maximum number of reuse of the path that it represents in $R$. Note that we treat each component of the RBS as an NCA relative to the non-RBS components, $eRU(u',v')$ must be greater than its reusability, e.g. greater than the PCAs of the cycles $(u',v')$ are involved in, hence, we add 1 to this maximum number of reuse. Additionally, note the $eRU$ of the arcs along the path $p$ can vary because of the reuse of such components within the RBS, albeit the number of times they are accessible through the in-bridges of their ancestor node is the same. However, the reusability of the entire path $p$ itself is bound to the minimum of the $L$-values of the arcs therein. Thus, this minimum shall be the representative $L$-value for these arcs as reflected by their representative abstract arc.
    \end{enumerate}
\end{algorithm}


\section*{Structural Verification of Weak Soundness of an RDLT}
\begin{algorithm}[H]
    \caption{Weak RDLT Soundness Verification Algorithm (WRSVA) }
    \label{WeakAlg}
    \begin{algorithmic}
        \State \textbf{Input:} RDLT $ R $ with or without RBS
        \State \textbf{Output:} True, false otherwise
    \end{algorithmic}
    \begin{algorithmic}[1]
        \State Apply Expanded Vertex Simplification Algorithm \cite{MalinaoWCTP2023}
        \If{$ R $ contains an RBS}
            \State $ i $ $ = $ $ 2 $
        \Else
            \State $ i $ $ = $ $ 1 $
        \EndIf
        \For{every vertex simplification $ R_i $}
            \For{every vertex $ v $ $ \in $ $ V $}
                \State Store deadlock point $ y $
            \EndFor
            \For{every deadlock point $ y $ $ \in $ $ V $}
                \State Determine if a non-critical loop-safe escape contraction path exists from $ y $
                \State Determine if the split-join pair containing $ y $ has weakened JOIN-safe L-values
            \EndFor            
        \EndFor
        \If{$ R_1 $ is deadlock-tolerant}
            \If{$ R $ contains an RBS}
                \If{$ R_2 $ deadlock-tolerant}
                    \State \Return true
                \EndIf
            \EndIf
            \State \Return true
        \Else
            \State \Return false
        \EndIf
    \end{algorithmic}
\end{algorithm}

\begin{thm}\textbf{Time Complexity of WRSVA}
    The time complexity of verifying that an RDLT $ R $ is weak-sound is $ O(c|E|^4) $.
    \label{TCWRSVA}
\end{thm}
\begin{proof}
    The algorithm is divided into three main processes: (1) expanded vertex simplification \cite{MalinaoWCTP2023}, (2) determining the deadlock-points, and (3) verifying deadlock tolerance. For the first process, it has a time complexity of $ O(|V|^2)$ which corresponds to the maximum number of arcs $ R $ can have. Because the algorithm visits every arc of the diagram to replicate them or create abstract versions for the outputs $ R_1 $ and $ R_2 $ (if an RBS exists), the worst-case scenario for this algorithm is when the RDLT has the maximum amount of arcs, hence $ O(|V|^2) $. 
    
    For the second process, it has a time complexity of $ O(|V|+|E|)$, where it corresponds to the sum of the number of vertices $ V $ and arcs $ E $ in $ R $. This process uses the depth-first or breadth-first search algorithm to solve the problem, hence $ O(|V|+|E|) $.

    The third process of verifying deadlock tolerance can be divided into its four requirements which has their own processes: (1) verifying loop-safeness, which takes $ O(c|E|^4) $ time \cite{MalinaoPJS2023}, (2) verifying existence of safe critical arcs, which takes $ O(|E|^2) $ time \cite{MalinaoPJS2023}, (3) verifying weakened JOIN-safeness, and (4) verifying deadlock resolution.
    
    The verification of weakened JOIN-safeness can be further divided into its component processes and determining each of their time complexities. Specifically for every AND- and MIX-JOIN, it concerns the processes of (1) determining the one split origin, (2) determining if there are no unrelated process, (3) determining if there are no branches, (4) determining if there are no processes interruptions with a C-value of $ Sigma $, (5) determining duplicate conditions at the merge point, (6) determining the equality of the L-values, and (7) determining the non-existence of critical arcs. The first three processes take $ O(|V||E|^2) $ time \cite{MalinaoPJS2023}. The fourth process takes $ O(|V||E|) $ time, similar to its stricter counterpart in \cite{MalinaoPJS2023} where it avoids all process interruptions no matter the $ C $-attribute. The fifth and sixth processes takes $ O(|V||E|log|E|) $ and $ O(|V||E|) $ time respectively \cite{MalinaoPJS2023}. The last process to determine the number takes $ O(c|E|) $ where $ c $ is the number of cycles as it is the highest time complexity of the required steps to find so, specifically Problem 1 to 3 in Lemma 1 of \cite{Malinao2017}. Because the dominating processes in terms of time complexity are the first three processes, the time complexity of verifying weakened JOIN-safeness is $ O(|V||E|^2) $.

    Lastly, verifying if the RDLT $ R $ is deadlock-resolving requires $ O(p(|V| + |E|)) $ where $ p $ signifies the number of deadlock points since deadlock-resolving is essentially the process of graph contraction, which takes $ O(|V| + |E|) $ \cite{MalinaoPJS2023}, for every deadlock point in $ R $.
    
    With the time complexities of the sub-processes of deadlock tolerance, verifying loop-safeness has the greatest complexity out of the processes, making the time complexity of determining deadlock tolerance is $ O(c|E|^4) $.

    Since the time complexity of the determining deadlock tolerance is greater out of the processes as well, the weak soundness of $ R $ can be determined in $ O(c|E|^4) $ time.
\end{proof}

\begin{thm}\textbf{Space Complexity of WRSVA}
    The space complexity of verifying that an RDLT $ R $ is weak-sound is $ O(|E|^2) $.
    \label{SCWRSVA}
\end{thm}
\begin{proof}
    As mentioned earlier, the algorithm is divided into three main processes. For the first process of EVS, it has a space complexity of $ O(|V|^2)$ which corresponds to the maximum number of arcs $ R $ can have. Because the algorithm stores every arc of the diagram to replicate them or create abstract versions for the outputs $ R_1 $ and $ R_2 $ (if an RBS exists), the worst-case scenario for this algorithm is when the RDLT has the maximum amount of arcs, hence $ O(|V|^2) $. 
    
    For the second process, it has a space complexity of $ O(|V|+|E|)$ similar to its time complexity. It stores the vertices and arcs traversed to create the contraction path as it uses the depth-first or breadth-first search algorithm to solve the problem, hence $ O(|V|+|E|) $.

    The third process of verifying deadlock tolerance can be divided into its four requirements which has their own processes: (1) verifying loop-safeness, which takes $ O(c|E|) $ space \cite{MalinaoPJS2023}, (2) verifying existence of safe critical arcs, which takes $ O(|v|+|E|) $ space \cite{MalinaoPJS2023}, (3) verifying weakened JOIN-safeness, and (4) verifying deadlock resolution.
    
    The verification of weakened JOIN-safeness can be further divided into its component processes and determining each of their space complexities. Specifically for every AND- and MIX-JOIN, it concerns the processes of (1) determining the one split origin, (2) determining if there are no unrelated process, (3) determining if there are no branches, (4) determining if there are no processes interruptions with a C-value of $ Sigma $, (5) determining duplicate conditions at the merge point, (6) determining the equality of the L-values, and (7) determining the non-existence of critical arcs. The first five processes as well as the last process use $ O(|E|^2) $ space \cite{MalinaoPJS2023}. However, the sixth process uses $ O(|V||E|) $ space. Because the sixth process needs the least amount of space and is the sole process with a different complexity, the space complexity of verifying weakened JOIN-safeness is $ O(|E|^2) $.

    Lastly, verifying if the RDLT $ R $ is deadlock-resolving requires $ O(p(|V| + |E|)) $ space similar to its time needed as this process is similar to the graph contraction for every deadlock point in $ R $.
    
    With the space complexities of the sub-processes of deadlock tolerance, verifying loop-safeness has the greatest complexity out of the processes, making the time complexity of determining deadlock tolerance is $ O(|E|^2) $.

    Since the space complexity of the determining deadlock tolerance is greater out of the processes as well, the weak soundness of $ R $ can be determined with $ O(|E|^2) $ space.
\end{proof}

\section*{Structural Verification of Easy Soundness of an RDLT}

\begin{thm}\textbf{Structural Verification of the Easy Soundness of an RDLT $ R $}
    \label{SVEasy}
    An RDLT $ R $ is easy-sound if and only if the following elements exist: 
    \begin{itemize}
        \item A contraction path $ P_1 $ in the level-1 vertex simplification $ R_1 $ of $ R $ from the initial vertex $ x_i $ to the final vertex $ x_f $, wherein $ P_1 $ $ = $ $ x_1, \ldots, x_f $
        \item A contraction path $ P_2 $ in the level-2 vertex simplification $ R_2 $ of $ R $ from the initial vertex $ z $ to the final vertex $ p $ of the RBS, wherein $ P_2 $ $ = $ $ z, \ldots, p $
        \item An in-bridge $ (x,y) $ formed by a component in $ P_1 $ such that there exists a component $ (y,z) $ in $ P_2 $
        \item An out-bridge $ (q,p) $ formed by a component in $ P_2 $ such that there exists a component $ (p,r) $ in $ P_1 $
    \end{itemize}
\end{thm}

With Theorem \ref{SVEasy} as the basis, Algorithm \ref{EasyAlg} verifies the easy soundness of a given input RDLT $ R $ through the satisfaction of a contraction path existing from the source to the sink vertex.

\begin{algorithm}[H]
    \caption{Easy RDLT Soundness Verification Algorithm (ERSVA) }
    \label{EasyAlg}
    \begin{algorithmic}
        \State \textbf{Input:} RDLT $ R $ with or without RBS
        \State \textbf{Output:} True, false otherwise
    \end{algorithmic}
    \begin{algorithmic}[1]
        \State Apply Vertex Simplification Algorithm \cite{Malinao2017}
        \If{$ R $ contains an RBS}
            \State $ i $ $ = $ $ 2 $
        \Else
            \State $ i $ $ = $ $ 1 $
        \EndIf
        \For{every vertex simplification $ R_i $}
            \State Apply Graph Contraction Strategy \cite{Malinao2017}
        \EndFor
        \If{$ R_1 $ has a contraction path from the source $ x_1 $ to the sink $ x_n $}
            \If{$ R $ contains an RBS}
                \If{$ R_2 $ has a contraction path from the source $ z $ to the sink $ p $}
                    \State \Return true
                \EndIf
            \EndIf
            \State \Return true
        \Else
            \State \Return false
        \EndIf
    \end{algorithmic}
\end{algorithm}

\begin{thm}\textbf{Time Complexity of ERSVA}
    The time complexity of verifying that an RDLT $ R $ is easy-sound is $ O(|V|^2) $.
    \label{TCERSVA}
\end{thm}
\begin{proof}
    The algorithm is divided into two main processes: (1) vertex simplification \cite{Malinao2017} and (2) graph contraction \cite{Malinao2017, MalinaoPJS2023}. For the first process, it has a time complexity of $ O(|V|^2)$ which corresponds to the maximum number of arcs $ R $ can have. Because the algorithm visits every arc of the diagram to replicate them or create abstract versions for the outputs $ R_1 $ and $ R_2 $ (if an RBS exists), the worst-case scenario for this algorithm is when the RDLT has the maximum amount of arcs, hence $ O(|V|^2) $. 
    
    For the second process, it has a time complexity of $ O(|V|+|E|)$, where it corresponds to the sum of the number of vertices $ V $ and arcs $ E $ in $ R $, because the process essentially determines if there exists a path from the source vertex to every vertex. Although this process is only done to make sure that there is a path from the source to the sink vertex, this complexity still applies as a worst-case scenario when there are no deadlocks within $ R $, i.e. $ R $ is live. As proved in \cite{MalinaoPJS2023}, this process uses the depth-first or breadth-first search algorithm to solve the problem, hence $ O(|V|+|E|) $.

    Since the time complexity of the vertex simplification is greater out of the two processes, the easy soundness of $ R $ can be determined in $ O(|V|^2) $ time.
\end{proof}

\begin{thm}\textbf{Space Complexity of ERSVA}
    \label{SCERSVA}
    The space complexity of verifying that an RDLT $ R $ is easy-sound is $ O(|V|^2) $.
\end{thm}
\begin{proof}
    Similar to the time complexity, the algorithm is divided into the vertex simplification and graph contraction. 
    
    For the first process, it has a space complexity of $ O(|V|^2)$ as the process stores every arc within $ R $ to generate $ R_1 $ and $ R_2 $ if $ R $ has an RBS. The worst-case scenario is $ R $ having the maximum amount of arcs given a number of vertices $ V $,  hence $ O(|V|^2) $. 

    For the second process, it has a space complexity of $ O(|V| + |E|) $ \cite{MalinaoPJS2023} as the process stores the vertices and arcs to simulate the possible contraction path of both the vertex simplifications. 

    Since the space complexity of the vertex simplification is greater out of the two processes, the easy soundness of $ R $ can be determined using $ O(|V|^2) $ space.
\end{proof}


% ------------------------------------------------------------------------
\INPUT{biblio.bib} 

\setlinespacing{1.44}
%\bibliographystyle{amsplain}
\bibliographystyle{plain}
\bibliography{biblio}
\end{document}
% ------------------------------------------------------------------------
